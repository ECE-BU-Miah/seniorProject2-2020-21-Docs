%======================================================================
\chapter{Conclusion and Future Work}
\label{ch: Chapter6}
\label{ch: conclusionAndFuture}
%======================================================================

%----------------------------------------------------------------------
\section{Conclusion}
%----------------------------------------------------------------------

This project shows that it is possible to use signal strength alone to make a mobile cart follow a remote target. A functional design for a mobile cart was developed and tested. In these tests, it was concluded that with the robot's current design, it would be extremely difficult to calculate an accurate distance to the remote from the signal strength alone. Therefore, the current design follows the target remote based on angle measurements alone. Due to the limitation of not having a usable distance value, the current implementation is unable to dynamically adjust its behavior based on proximity to the remote. One such planned parameter was the robot's velocity, which would have been reduced as the robot approached the remote target to prevent it from running into the user. Based on the literature review, it was determined that the Smart Robotic Cart design from this project is more cost-effective than most, if not all, of the existing solutions. It was shown that the robot could successfully track the remote target even when obstacles were obstructing the line-of-sight path between the robot and remote.

%----------------------------------------------------------------------
\section{Future Work}
%----------------------------------------------------------------------

Although the robotic cart prototype did successfully follow the remote based on signal strength alone, there is still much to be desired. The first item that would need to be addressed is a proper obstacle avoidance system. For the current robotic cart prototype, it was initially planned to have the robot avoid collision with the user by using the distance between the robot and remote, but this was dropped due to the inaccuracy of the distance measurement. This user avoidance could be incorporated into an obstacle detection and avoidance system in a future project, most likely using infrared or sonar sensors mounted on the robot. This is a necessary continuation of this project since, as it currently stands, it can detect and follow the remote through obstacles but will attempt to drive in a straight path towards the remote despite the obstacle between them.

\vspace*{12pt}
\noindent
The project also requires more overall tuning in any future work that is done on it. The two main places that need more tuning are the weights used to normalize the signal strengths out of the reflector array and the specs of the reflector array's design to optimize its performance. The values for normalizing the signal strengths coming out of the array currently are based on a relatively small and limited set of recorded data from the robot. For a future extension of this project, more measurements would be required to better determine the quality of the data coming out of each reflector. These values would most likely be obtained by taking many measurements in an anechoic chamber, which was also used for the current smaller set of sensor data. The data currently is only being used to calculate a base offset for each reflector's strength, though with enough data, it may be possible to use the top radio module's strength to adjust for overall degradation to the signal strength. The reflectors themselves also most likely need to be redesigned again. During testing, it was noted that the difference from facing the target to facing away from the target is only slightly greater than the average signal strength measurement noise.

\vspace*{12pt}
\noindent
Another problem that needs to be addressed in this project is determining a reasonably accurate distance measurement. The problem with distance stems from the fact that the locations remote and robot are relative to each other, so there is no global reference point that can be used to stabilize the measurements. The second problem is that the only way with the signal strength to get distance is to use a version of free-space path loss, which fails when the user stands between the remote and the robot. For this, two possible solutions were thought up after it was determined that signal strength alone would not work. These solutions were also determined to be outside the scope of what could be reasonably achieved within the allotted project time. The two possible solutions are to use signal strength with multi-frequency radio modules or to use the travel time of the message from the robot. The multi-frequency approach would still use signal strength but would potentially allow for a profile to be put together from the signal strength of each frequency to try and ignore the interference from objects. The other approach of using a message's travel time has been shown to work in such applications as GPS over large distances. This approach may work since the speed of the radio signal is generally less affected by objects in its path than signal strength is. The main challenge with this approach would be getting a sufficiently accurate time measurement for short distances. If either of these approaches to determining distance ended up working, then the algorithm could be more dynamic in its behavior and interpretation of the signal strengths.

%%% Local Variables:
%%% mode: latex
%%% TeX-master: "../finalReport"
%%% End:
