% uOttawa (unofficial) Thesis Template for LaTeX 
% Edited by Wail Gueaieb based on Stephen Carr's uWaterloo Template

% The files included in this package are slighly modified by Suruz Miah to adapt partial requirements  in writing project/thesis reports of the Bradley University's Department of Electrical and Computer Engineering.

% DON'T USE THIS TEMPLATE IF YOU DON'T KNOW WHAT YOU'RE DOING!
% Remember, it comes WITH NO WARRANTY!

% Please read the "00readme.txt" file first.
% Here is how to use this template:
%
% DON'T FORGET TO ADD YOUR OWN NAME AND TITLE in the "hyperref" package
% configuration in the "thesis-preample.tex" file. THIS INFORMATION GETS 
% EMBEDDED IN THE PDF FINAL PDF DOCUMENT.
% You can view the information if you view Properties of the PDF document.

% The template is based on the standard "book" document class which provides 
% all necessary sectioning structures and allows multi-part theses.

% DISCLAIMER
% To the best of our knowledge, this template satisfies the current 
% uOttawa thesis requirements.
% However, it is your responsibility to assure that you have met all 
% requirements of the university and your particular department.
% Many thanks to the feedback from many graduates that assisted the 
% development of this template.

% -----------------------------------------------------------------------

% When using pdflatex, by default the output is geared toward generating a PDF 
% version optimized for viewing on an electronic display, including 
% hyperlinks within the PDF.
 
% E.g. to process a thesis based on this template, run:

% (pdf)latex thesisMain	-- first pass of the (pdf)latex processor
% bibtex thesisMain 	-- generates bibliography from .bib data file(s) 
% (pdf)latex thesisMain	-- fixes cross-references, bibliographic references, etc
% (pdf)latex thesisMain	-- fixes cross-references, bibliographic references, etc
% makeindex -s nomentbl.ist -o thesisMain.nls thesisMain.nlo
% (pdf)latex thesisMain	-- fixes cross-references, bibliographic references, etc
% (pdf)latex thesisMain	-- fixes cross-references, bibliographic references, etc



% N.B. The "pdftex" program allows graphics in the following formats to be
% included with the "\includegraphics" command: PNG, PDF, JPEG, TIFF
% Tip 1: Generate your figures and photos in the size you want them to appear
% in your thesis, rather than scaling them with \includegraphics options.
% Tip 2: Any drawings you do should be in scalable vector graphic formats:
% SVG, PNG, WMF, EPS and then converted to PNG or PDF, so they are scalable in
% the final PDF as well.
% Tip 3: Photographs should be cropped and compressed so as not to be too large.

% To create a PDF output that is optimized for double-sided printing: 
%
% 1) comment-out the \documentclass statement in the preamble below, and
% un-comment the second \documentclass line.
%
% 2) change the value assigned below to the boolean variable
% "PrintVersion" from "false" to "true".

% --------------------- Start of Document Preamble -----------------------

% Specify the document class, default style attributes, and page dimensions
% For hyperlinked PDF, suitable for viewing on a computer, use this:
% \UseRawInputEncoding

% \documentclass[letterpaper,12pt,titlepage,oneside,final]{book}
 
% For PDF, suitable for double-sided printing, change the PrintVersion variable below
% to "true" and use this \documentclass line instead of the one above:
\documentclass[letterpaper,12pt,titlepage,openright,twoside,final]{book}


% This package allows if-then-else control structures.
\usepackage{ifthen}
\newboolean{PrintVersion}
\setboolean{PrintVersion}{false} 
% \setboolean{PrintVersion}{true} 
% CHANGE THIS VALUE TO "true" as necessary, to improve printed results 
% for hard copies by overriding some options of the hyperref package.

%%%%%%%%%%%%%%%%%%%%%
% MATLAB Code
\usepackage[framed,numbered,autolinebreaks,useliterate]{mcode}
%%%%%%%%%%%%%%%%%%%%%

%%%%%%%%%%%%%%%%%%%%%
% Algorithm 
\usepackage[english,algo2e,algoruled,vlined,linesnumbered]{algorithm2e}
%%%%%%%%%%%%%%%%%%%%%

%%%%%%%%%%%%%%%%%%%%%
% Table
\usepackage{booktabs}
%%%%%%%%%%%%%%%%%%%%%

%%%%%%%%%%%%%%%%%%%%%
% Enable Subfigures
\usepackage{subfigure}
%%%%%%%%%%%%%%%%%%%%%

%%%%%%%%%%%%%%%%%%%%%
% Enable todonotes
\usepackage{todonotes}
%%%%%%%%%%%%%%%%%%%%%

% Load your needed packages and other commands of yours.
% Load your needed packages and other commands of yours here:
%\usepackage{} % ... note that old .sty files can be included here

















%--------------------------------------------------------------------------
% Do NOT edit the rest of the preample UNLESS YOU KNOW WHAT YOU'RE DOING!
%--------------------------------------------------------------------------

\ifthenelse{\boolean{PrintVersion}}{
\usepackage[top=1in,bottom=1in,left=0.75in,right=1.25in]{geometry}   % For twoside document
}{
\usepackage[top=1in,bottom=1in,left=0.75in,right=1.25in]{geometry}   % For oneside document
}

\usepackage{amsmath,amssymb,amstext} % Lots of math symbols and environments
\usepackage{graphicx} % For including graphics 

\usepackage{nomentbl} 
\makenomenclature 

\usepackage{ifpdf}

\newcommand{\href}[1]{#1} % does nothing, but defines the command so the
    % print-optimized version will ignore \href tags (redefined by hyperref pkg).
%\newcommand{\texorpdfstring}[2]{#1} % does nothing, but defines the command
% Anything defined here may be redefined by packages added below...


% Hyperlinks make it very easy to navigate an electronic document.
% In addition, this is where you should specify the thesis title
% and author as they appear in the properties of the PDF document.
% Use the "hyperref" package 
% N.B. HYPERREF MUST BE THE LAST PACKAGE LOADED; ADD ADDITIONAL PKGS ABOVE
\usepackage[\ifpdf pdftex,\fi letterpaper=true,pagebackref=false]{hyperref} % with basic options
		% N.B. pagebackref=true provides links back from the References to the body text. This can cause trouble for printing.
\hypersetup{
    plainpages=false,       % needed if Roman numbers in frontpages
    pdfpagelabels=true,     % adds page number as label in Acrobat's page count
    bookmarks=true,         % show bookmarks bar?
    unicode=false,          % non-Latin characters in Acrobat's bookmarks
    pdftoolbar=true,        % show Acrobat's toolbar?
    pdfmenubar=true,        % show Acrobat's menu?
    pdffitwindow=false,     % window fit to page when opened
    pdfstartview={FitH},    % fits the width of the page to the window
%    pdftitle={uOttawa\ LaTeX\ Thesis\ Template},    % title: CHANGE THIS TEXT!
%    pdfauthor={Author},    % author: CHANGE THIS TEXT! and uncomment this line
%    pdfsubject={Subject},  % subject: CHANGE THIS TEXT! and uncomment this line
%    pdfkeywords={keyword1} {key2} {key3}, % list of keywords, and uncomment this line if desired
    pdfnewwindow=true,      % links in new window
    colorlinks=true,        % false: boxed links; true: colored links
    linkcolor=blue,         % color of internal links
    citecolor=green,        % color of links to bibliography
    filecolor=magenta,      % color of file links
    urlcolor=cyan           % color of external links
}
\ifthenelse{\boolean{PrintVersion}}{   % for improved print quality, change some hyperref options
\hypersetup{	% override some previously defined hyperref options
%    colorlinks,%
    citecolor=black,%
    filecolor=black,%
    linkcolor=black,%
    urlcolor=black}
}{} % end of ifthenelse (no else)

\usepackage{fancyhdr,lastpage} % Change caption style; changes headers and page styles etc.
\usepackage{epstopdf}
\epstopdfsetup{suffix={}}


% This is where thesis margins and spaces are set.
\input{private/thesis-margins-and-spaces}

\fancypagestyle{myFancy}{%
  \fancyhf{}% Clear header and footer
  \fancyhead[LE,RO]{\bfseries\nouppercase{\rightmark}}
  \fancyhead[LO,RE]{\bfseries\nouppercase{\leftmark}}
  \fancyfoot[R]{Page \thepage\ of \pageref{LastPage}}% Custom footer
  \fancyfoot[L]{G.~Janiak \& K.~Vonckx (\nameOfUniversity)}% Custom footer
  \renewcommand{\headrulewidth}{0.4pt}% Line at the header visible
  \renewcommand{\footrulewidth}{0.1pt}% Line at the footer visible
}


%======================================================================
%   L O G I C A L    D O C U M E N T -- the content of your thesis
%======================================================================
\begin{document}

% For a large document, it is a good idea to divide your thesis
% into several files, each one containing one chapter.
% To illustrate this idea, the "front pages" (i.e., title page,
% declaration, borrowers' page, abstract, acknowledgements,
% dedication, table of contents, list of tables, list of figures,
% nomenclature).
%----------------------------------------------------------------------
% FRONT MATERIAL
%----------------------------------------------------------------------
%
% C O V E R  P A G E
% ------------------
\newcommand{\thesisauthor}{Glenn Janiak and Ken Vonckx}
\newcommand{\advisor}{Dr. Suruz Miah}
\newcommand{\thesistitlecoverpage}{%
Smart Control of 2-Degree of Freedom Helicopters 
}
%\newcommand{\degree}{Ph.D.} % possible values are:
                            % M.A. / M.A.Sc. / M.Sc. / MCS / Ph.D.
\newcommand{\nameofprogram}{Electrical and Computer Engineering Department}
\newcommand{\academicunit}{Caterpillar College of Engineering and Technology}
%\newcommand{\faculty}{Faculty of Engineering}
\newcommand{\nameOfUniversity}{Bradley University}
\newcommand{\graduationyear}{2019}
%
% T I T L E   P A G E
% -------------------
% Last updated May 24, 2011, by Stephen Carr, IST-Client Services
% The title page is counted as page `i' but we need to suppress the
% page number.  We also don't want any headers or footers.
\pagestyle{empty}
\pagenumbering{roman}

% The contents of the title page are specified in the "titlepage"
% environment.
\begin{titlepage}
        \begin{center}
        \vspace*{1.0cm}

        \Huge
        {\bf \thesistitlecoverpage }

        \vspace*{1.0cm}

        \normalsize
        by \\

        \vspace*{1.0cm}

        \Large
        \thesisauthor\\
        Advisors:~\advisor\\

        \vspace*{3.0cm}

        % \normalsize
        % Thesis submitted to the\\
        % Faculty of Graduate and Postdoctoral Studies\\
        % In partial fulfillment of the requirements\\
        % For the \degree~degree in\\
        % \nameofprogram\\

        \vspace*{2.0cm}

        \nameofprogram\\
        \academicunit\\
        %\faculty\\
        \nameOfUniversity\\

        \vspace*{4.0cm}

        \copyright~\thesisauthor\\Peoria, Illinois, \graduationyear\\
        \end{center}
\end{titlepage}

% The rest of the front pages should contain no headers and be numbered using Roman numerals starting with `ii'
% PRELIMINARY PAGES

\pagestyle{plain}
\setcounter{page}{2}

\cleardoublepage % Ends the current page and causes all figures and tables that have so far appeared in the input to be printed.
% In a two-sided printing style, it also makes the next page a right-hand (odd-numbered) page, producing a blank page if necessary.



%%% Local Variables:
%%% mode: latex
%%% TeX-master: "../finalReport"
%%% End:




%
% R E S T  O F  F R O N T  P A G E S
% ----------------------------------
% % D E C L A R A T I O N   P A G E
% -------------------------------
  % This page is not needed for a uOttawa thesis. Don't include it.
  % It is designed for an electronic thesis.
  \noindent
I hereby declare that I am the sole author of this thesis. This is a true copy of the thesis, including any required final revisions, as accepted by my examiners.

  \bigskip
  
  \noindent
I understand that my thesis may be made electronically available to the public.

\cleardoublepage
%\newpage
 %This is not needed in a uOttawa thesis.
%
% Edit the following 3 files with your abstract, acknowledgements, 
% and dedication.
% A B S T R A C T
% ---------------

\begin{center}\textbf{Abstract}\end{center}

%This paper proposes a strategy for testing and comparing three control algorithms, LQR, LQG, and ADP, to control two two-DOF helicopters from a mobile device.  We will be using Raspberry Pi 3's as terminals for the wireless communication and MATLAB as our primary coding language.

This project proposes a modular and cost-effective smart real-time motion control
framework for a group of two degrees of freedom (2-DOF) helicopters.  The helicopters are
controlled by a mobile device which sends position signals over a wireless network to a
microcontroller.  The microcontroller uses control algorithms to determine the amount of voltage
to apply to DC motors.  This project tests and compares three control algorithms, LQR, LQG, and
ADP as well as theorizes possible methods to improve the control in future projects.

\cleardoublepage
%\newpage


%%% Local Variables:
%%% mode: latex
%%% TeX-master: "../finalReportMainV1"
%%% End:

% A C K N O W L E D G E M E N T S
% -------------------------------

\begin{center}\textbf{Acknowledgements}\end{center}

    \par Special Thanks to Andrew Fandel, Anthony Birge, and Dr. Suruz Miah for their work with Machine Learning on a 2-DOF Helicopter.
    \par Thanks to Mr. Christopher Mattus for his assistance in setting up our lab environment.
    \par Thank you to everyone else who helped make this project possible and sucessful.


\cleardoublepage
%\newpage



%%% Local Variables:
%%% mode: latex
%%% TeX-master: "../finalReportMainV1"
%%% End:

% D E D I C A T I O N
% -------------------

\begin{center}\textbf{Dedication}\end{center}

This is dedicated to our loved ones and friends who supported us through our time working on this project.

\cleardoublepage
%\newpage


%%% Local Variables:
%%% mode: latex
%%% TeX-master: "../finalReportMainV1"
%%% End:

%
%
% No need to edit this file.
% T A B L E   O F   C O N T E N T S
% ---------------------------------
\renewcommand\contentsname{Table of Contents}
\tableofcontents
\cleardoublepage
\phantomsection
%\newpage

% L I S T   O F   T A B L E S
% ---------------------------
\addcontentsline{toc}{chapter}{List of Tables}
\listoftables
\cleardoublepage
\phantomsection		% allows hyperref to link to the correct page
%\newpage

% L I S T   O F   F I G U R E S
% -----------------------------
\addcontentsline{toc}{chapter}{List of Figures}
\listoffigures
\cleardoublepage
\phantomsection		% allows hyperref to link to the correct page
%\newpage


%
% No need to edit this file. But you may want to comment the whole line if you
% don't have or want a Nomenclature section.
% L I S T   O F   S Y M B O L S
% -----------------------------
% To include a Nomenclature section
\addcontentsline{toc}{chapter}{\textbf{Nomenclature}}

\renewcommand{\nomname}{Nomenclature}
\renewcommand{\nomAname}{\textbf{\large Abbreviations}}
\renewcommand{\nomGname}{\textbf{\large Mathematical Symbols}}
\renewcommand{\nomXname}{\textbf{\large Superscripts}}
\renewcommand{\nomZname}{\textbf{\large Subscripts}}

\printnomenclature
\cleardoublepage
\phantomsection % allows hyperref to link to the correct page
% \newpage


\nomAname
\begin{itemize}
    \item[]\textbf{RF} - Radiofrequency
\end{itemize}
\bigbreak

\nomGname
\begin{itemize}
    \item[]$K_v$ - proportional gain for linear velocity
    \item[]$K_\omega$ - proportional gain for angular velocity
    \item[]$\theta_r$ - angle of the remote with respect to robot
    \item[]$d_r$ - distance from robot to remote
\end{itemize}


%%% Local Variables: 
%%% mode: latex
%%% TeX-master: "../uottawa-thesis"
%%% End:   


% Change page numbering back to Arabic numerals
\pagenumbering{arabic}



%

% Redefine the plain page style
\fancypagestyle{plain}{%
  \fancyhf{}%
  \fancyfoot[R]{Page \thepage\ of \pageref{LastPage}}%
  \fancyfoot[L]{G.~Janiak \& K.~Vonckx (\nameOfUniversity)}%  
  \renewcommand{\headrulewidth}{0pt}% Line at the header invisible
  \renewcommand{\footrulewidth}{0.1pt}% Line at the footer visible
}
\pagestyle{myFancy}


%----------------------------------------------------------------------
% MAIN BODY
%---------------------------------------------------------------------- 
% Chapters 
% Include your "sub" source files here (must have extension .tex)
%======================================================================
\chapter{Introduction}
\label{ch: Chapter1}
%======================================================================

%----------------------------------------------------------------------
\section{Problem Statement}
%----------------------------------------------------------------------
Helicopters are of a paramount importance as
they are used in many civilian and military applications due to their ability for vertical take-off and landing. To enable their use in such applications, intensive research has been conducted to date since helicopters involve complex nonlinear dynamics. Most of the work on helicopter based research requires dedicated computers for controlling their motion to specific configurations and resistant to turbulent conditions. Such methods are expensive and time-consuming to develop. Implementation of motion control techniques using cost-effective hardware is still a challenge.

In this project, we are proposing an algorithm for smart control of a team of two degree-of freedom (two-DOF) helicopters using conventional motion control in cooperation with machine learning techniques where a user will be able to configure helicopters from any initial position. Even though conventional techniques have been tested with simple platforms in the literature, the current project employs conventional motion control strategies in cooperation with machine learning technique (reinforcement learning, for instance) for a team of helicopters as well as introducing user control via mobile devices. This project is expected to encourage research in this area as well as serve as an educational tool in teaching environments.


%----------------------------------------------------------------------
\section{Literature Review}
%----------------------------------------------------------------------
Our project requires a great deal of research as some of our tasks have not been attempted before.  As a result, we have examined research papers, work complete by other projects at Bradley University, and documentation/teaching materials from Quanser Inc.

Among the major challenges in developing unmanned systems is to implement a modular, cost-effective, and robust teleoperation system, where the motion of a group of helicopters is controlled by mobile devices. In some cases, computer simulations are conducted to reveal the performance control structures of a 2-DOF helicopter~\cite{DArpino2010}. A large body of research has been conducted in the literature to focus on developing different control structures, such as linear-quadratic regulators (LQR)/Gaussian (LQG), sliding-mode controls (SMC), and advance nonlinear controls, that are specifically applied to 2-DOF helicopters. See~\cite{Subramanian2016-Robust,Ahmed2010-Sliding}, for example, and some references therein. Furthermore,  soft-computing tools, such as fuzzy-logic, neural networks, and a few combinations of them are employed for controlling the motion of a 2-DOF helicopter~\cite{Chang2017-Fuzzy,Kayacan2016-Fuzzy,Gao2016-DataDriven,Hernandez2012-Decentralized}.

The documentation of Quanser AERO\footnote{See \href{https://www.quanser.com/}{https://www.quanser.com/} for details} employs LQR and LQG motion control techniques for teaching purposes.  This involves creating a linearized system model to calculate the LQR state-feedback gain.  A model reference adaptive control (MARC) scheme using Lyapunov functions has been used by~\cite{Subramanian2016-Robust} for adaptive motion control of a 2-DOF helicopter.  SMC is a nonlinear control technique to drive the system states onto a surface in the state space.  This method has been used  by~\cite{Ahmed2010-Sliding}.  Fuzzy-logic controllers use an inference engine to produce an output as used in~\cite{Chang2017-Fuzzy,Kayacan2016-Fuzzy}.  The ADP technique does not rely on knowledge of the system model.  Instead, it uses data to reconstruct the states as preformed by~\cite{Gao2016-DataDriven}.  Authors in~\cite{Hernandez2012-Decentralized} used high order neural networks (HONN) to approximate non-linearities in the system model.

As can be noticed, most of the motion control techniques are either tested using computer simulations or use dedicated computational platforms, that may not be modular and/or cost-effective, for developing motion control algorithms. This is due to the fact that the aforementioned techniques are mainly to propose novel motion control techniques and not focused on hardware implementation platforms. Therefore, the current work considers implementing a conventional motion control technique using modular and cost-effective hardware platforms in the context of a teleoperation system, where a human operator has the ability to control the motion of a team of helicopters using a smart mobile device.


%----------------------------------------------------------------------
\section{Report Organization}
%----------------------------------------------------------------------
This report is organised into 6 chapters.  Chapter \ref{ch: Chapter1} discusses what this project hopes to accomplish and what similar projects have completed.  Chapter \ref{ch: Chapter2} explains how to create a mathematical model for the system.  Chapter \ref{ch: Chapter3} provides a breif explanation of the control algorithms and architectures used.  Chapter \ref{ch: Chapter4} contains the results from our simulations.  Chapter \ref{ch: Chapter5} disucsses results from USB, Raspberry Pi, and Mobile Device implementation.  Chapter \ref{ch: Chapter6} concludes the paper and provides insight into future directions.

%%% Local Variables:
%%% mode: latex
%%% TeX-master: "../finalReportMainV1"
%%% End:
% \input{parts/20-modeling2-DOF.tex}
% \input{parts/30-controlAlgorithms.tex}
% \input{parts/40-simulations.tex}
% %======================================================================
\chapter{Implementation}
\label{ch: Chapter5}
%======================================================================

%----------------------------------------------------------------------
\section{Robot Assembly}
%----------------------------------------------------------------------

For this project we needed a Remote Target and Smart Robotic Cart that we designed and implemented from off the shelf components.  As mentioned above in \autoref{sec:System Components} some of our components came from what the school already owned as well as some we had to buy specifically for this project listed in \autoref{tab:Partslablist} and \autoref{tab:Partslablist}.\par
Since the school already owned the Budget Bot chassis, we opted to use these base cart chassis frames to build our Smart Robotic Cart upon.  The base Budget Bot chassis did need to be modified slightly to work with our project.  The main change was that we swapped out the motors that came with the Budget Bot for Pololu 27D Meatal Gear motors since the original motors spin at max speed is 212 RPM or 1.09m/s and the new motors max speed is 530 RPM or 2.72m/s when using the base wheels that have a diameter of 98mm.  By switching out the motors we are able to achieve our goal of  matching an average person's walking speed of 1.5m/s since the build in motors can not achieve this. The other modifications to the chassis are that a hard power switch was added to directly cut off power to the XBee modules in the reflector as well as a power indicator LED. \par

\begin{figure}
	\centering
	\includegraphics[width=0.5\textwidth]{figs/img/Finalized_robot.png}
	\caption{Final Version of the Robotic Cart}
 	\label{fig:FinalizedRobot}
\end{figure}

As can be seen in \autoref{fig:FinalizedRobot} our next step was to attach the Beagle Bone Blue microcomputer, XBee reflector array assembly, and breadboard to the top of the robot.  We also installed a 2-cell LiPo battery inside the body of the mobile cart that delivers power to the Beagle Bone Blue directly and power to the breadboard through the switch. \par
The power supplied to the breadboard is sent through a regulator circuit to drop it down from 7.4v from the LiPo to 3.3v.  This same circuit also is used on the remote which consists only of a 9v battery and the XBee module with this voltage regulator in between.  This voltage regulator is build out of a LM117 regulator and a 10uF ceramic capacitor between the input and output pins of the regulator as shown in \autoref{fig:PowerConverter}.

\begin{figure}
	\centering
	\includegraphics[width=0.5\textwidth]{figs/img/PowerConverter.png}
	\caption{Final Version of the Robotic Cart}
 	\label{fig:PowerConverter}
\end{figure}

The Reflector array assembly from \autoref{ch: Chapter3} was mounted on a stepper motor that was then attached to the robot chasse using a 3D printed bracket.  This bracket was the attached to the chasse using bolts.  Another feature of this bracket is that it had a stop block built into the top of it that is used to automatically zero the reflector array when the robot starts up.\par
The Xbee modules from the reflector posed a bit of a problem when it came to connecting them to the Beagle Bone Blue.  This problem comes from that the Beagle Bone Blue has five UART ports if we are using the one found in the USB and the UART-GPS along with the three normal UART ports.  This unfortunately dose not work though since UART port zero is tied to the council used to communicate with it from a computer.  Since this port is reserved for this function that reduces the number of usable ports down to four which does not work since we have five XBees in it.  Our solution to this was to use two of the GPIO pins, two of the UART ports, and a two-way multiplexer.  With this setup we can directly connect one of the UARTs to the top XBee and then rout the other four side XBee modules through the MUX so only one UART port is needed for them.


%----------------------------------------------------------------------
\section{Code Base}
%----------------------------------------------------------------------

For the project we needed some base functionality to use with the main follower program so it can interact with the other components in the system.  The Beagle Bone Blue already has a set of ports on it that we can access and control using the Robot Control Library from Strawson Design for the beagle bone blue ~\cite{Robot_Control_Library}. \par
For the project we also build some custom libraries on top of the Robot Control Library to hand XBee Frames, AT Commands, and the stepper motor.  The XBee frames library takes a given block of data containing our message and then packages it into a XBee Frame to send over UART to the XBee module we want to interact with.  The XBee communications library also handles received messages from the XBee module and validates the received frames integrity and extracts the message data from it.  On top of the XBee communications library we also have a library that generates AT Commands to package in the XBee Frames.  The AT Command library also handles extracting response data from the AT Response Command structure and validating it.\par
The final custom library is the stepper motor control library.  This library had to be written ourselves since instead of using a dedicated stepper motor controller for our project we were using two of the regular DC motor control ports that are build into the Beagle Bone Blue.  This library handled translating how many steps to turn into proper cycling of the DC motor ports to get the stepper motor to move the desired distance.  The library also helped to keep track of basic information related to the stepper motor such as what its current angel should be.


%----------------------------------------------------------------------
\section{Algorithm Testing}
%----------------------------------------------------------------------



%----------------------------------------------------------------------
\section{Experimental Results}
%----------------------------------------------------------------------


\todo[inline]{See page 23 of~\url{http://ee.bradley.edu/projects/proj2017/ekf_slam/Downloads/seniorProject_finalReport.pdf}}


\section{Simulation using Robot Simulator}

\todo[inline]{In this section, we detail the steps of simulating the proposed
  robotic cart using a commercial robot simulator CoppeliaSim. $\ldots \ldots$}


%%% Local Variables:
%%% mode: latex
%%% TeX-master: "../finalReport"
%%% End:

% \input{parts/60-conclusion.tex}

% %----------------------------------------------------------------------
% % APPENDICES
% %---------------------------------------------------------------------- 
\appendix
% % Designate with \appendix declaration which just changes numbering style 
% % from here on
% % Add a title page before the appendices and a line in the Table of Contents
\addcontentsline{toc}{chapter}{APPENDICES} 
% %

% \input{parts/A-quanserParam.tex}
% \input{parts/B-simulation.tex}
% %\input{parts/C-USB.tex}
% \input{parts/D-RP.tex}
% \input{parts/E-ADP.tex}
% \input{parts/tutorial.tex}

% %----------------------------------------------------------------------
% % END MATERIAL
% %----------------------------------------------------------------------

% % B I B L I O G R A P H Y
% % -----------------------
% %
% % The following statement selects the style to use for references.  It controls the sort order of the entries in the bibliography and also the formatting for the in-text labels.
\bibliographystyle{plain}
% % This specifies the location of the file containing the bibliographic information.  
% % It assumes you're using BibTeX (if not, why not?).
% \ifthenelse{\boolean{PrintVersion}}{
% \cleardoublepage % This is needed if the book class is used, to place the anchor in the correct page,
%                  % because the bibliography will start on its own page.
% }{
% \clearpage       % Use \clearpage instead if the document class uses the "oneside" argument
% }
% \phantomsection  % With hyperref package, enables hyperlinking from the table of contents to bibliography             
% % The following statement causes the title "References" to be used for the bibliography section:
% % \renewcommand*{\bibname}{References}
% Bibliography 
\renewcommand{\bibname}{Bibliography}

% Add the References to the Table of Contents
\addcontentsline{toc}{chapter}{\textbf{References}}

\bibliography{bib/refsHelicopter,bib/refsSuruzWeb}
% Tip 5: You can create multiple .bib files to organize your references. 
% Just list them all in the \bibliogaphy command, separated by commas (no spaces).


%----------------------------------------------------------------------
\end{document}
%======================================================================



%%% Local Variables: 
%%% mode: latex
%%% TeX-master: t
%%% End: 
