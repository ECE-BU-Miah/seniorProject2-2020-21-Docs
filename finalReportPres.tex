% Copyright 2004 by Till Tantau <tantau@users.sourceforge.net>.
%
% In principle, this file can be redistributed and/or modified under
% the terms of the GNU Public License, version 2.
%
% However, this file is supposed to be a template to be modified
% for your own needs. For this reason, if you use this file as a
% template and not specifically distribute it as part of a another
% package/program, I grant the extra permission to freely copy and
% modify this file as you see fit and even to delete this copyright
% notice. 

% \UseRawInputEncoding
\documentclass{beamer}

% There are many different themes available for Beamer. A comprehensive
% list with examples is given here:
% http://deic.uab.es/~iblanes/beamer_gallery/index_by_theme.html
% You can uncomment the themes below if you would like to use a different
% one:
%\usetheme{AnnArbor}
%\usetheme{Antibes}
%\usetheme{Bergen}
%\usetheme{Berkeley}
%\usetheme{Berlin}
%\usetheme{Boadilla}
%\usetheme{boxes}
%\usetheme{CambridgeUS}
%\usetheme{Copenhagen}
%\usetheme{Darmstadt}
%\usetheme{default}
%\usetheme{Frankfurt}
%\usetheme{Goettingen}
%\usetheme{Hannover}
%\usetheme{Ilmenau}
%\usetheme{JuanLesPins}
%\usetheme{Luebeck}
\usetheme{Madrid}
%\usetheme{Malmoe}
%\usetheme{Marburg}
%\usetheme{Montpellier}
%\usetheme{PaloAlto}
%\usetheme{Pittsburgh}
%\usetheme{Rochester}
%\usetheme{Singapore}
%\usetheme{Szeged}
%\usetheme{Warsaw}

\usepackage{pgfgantt}
\usepackage{todonotes}
\usepackage{media9}
\usepackage{subfigure}
\usepackage{booktabs,array}


% Customize Warsaw color 
\setbeamercolor*{palette primary}{use=structure,fg=white,bg=red!50!black}
\setbeamercolor*{palette secondary}{use=structure,fg=white,bg=red!60!black}
\setbeamercolor*{palette tertiary}{use=structure,fg=white,bg=red!70!black}

% Customize Warsaw block title and background colors
\setbeamercolor{block title}{bg=red!50!black,fg=white}

\setbeamertemplate{bibliography item}{\insertbiblabel}  % insert bibliography numbers instead of symbol
\setbeamertemplate{caption}[numbered] % adds the figure or table number to the caption.



\title[2-DOF Helicopter]{Smart Control Algorithm for 2-DOF Helicopter}

% % A subtitle is optional and this may be deleted
% \subtitle{Product Proposal}

\author[G.~Janiak, K.~Vonckx]{Glenn~Janiak \and Kenneth~Vonckx \and
Advisor: Dr. Suruz Miah}
% - Give the names in the same order as the appear in the paper.
% - Use the \inst{?} command only if the authors have different
%   affiliation.

\institute[Bradley University] % (optional, but mostly needed)
{
  Department of Electrical and Computer Engineering\\
  Bradley University\\
  1501 W. Bradley Avenue\\
  Peoria, IL, 61625, USA
}
% - Use the \inst command only if there are several affiliations.
% - Keep it simple, no one is interested in your street address.

\date[May~4,~2019]{Saturday, May~4,~2019}

% - Either use conference name or its abbreviation.
% - Not really informative to the audience, more for people (including
%   yourself) who are reading the slides online

\logo{\hfill\href{http://www.bradley.edu}{\includegraphics[width=0.75cm]{figs/logoBU1-Print}}}  % place logo in every page 


\subject{Mobile Robot Localization}
% This is only inserted into the PDF information catalog. Can be left
% out. 

% If you have a file called "university-logo-filename.xxx", where xxx
% is a graphic format that can be processed by latex or pdflatex,
% resp., then you can add a logo as follows:

% \pgfdeclareimage[height=0.5cm]{university-logo}{university-logo-filename}
% \logo{\pgfuseimage{university-logo}}

% Delete this, if you do not want the table of contents to pop up at
% the beginning of each subsection:
\AtBeginSubsection[]
{
  \begin{frame}<beamer>{Outline}
    \tableofcontents[currentsection,currentsubsection]
  \end{frame}
}

% Let's get started
\begin{document}

\begin{frame}
  \titlepage
\end{frame}

\begin{frame}{Outline} 
  \tableofcontents%[pausesections]
  % You might wish to add the option [pausesections]
\end{frame}

% Section and subsections will appear in the presentation overview
% and table of contents.
\section{Introduction}

\begin{frame}{Introduction}{}
  % applications of mobile robot navigation and problem description
    \begin{itemize}
        \item Helicopter are important for short-distance travel
            \begin{itemize}
                \item air-sea rescue
                \item fire fighting
                \item traffic control
                \item tourism
            \end{itemize}
        \item Purpose of control system
            \begin{itemize}
                \item resistance to turbulence
                \item enable use of mobile device
            \end{itemize}
        \item Which is better?
            \begin{itemize}
                \item Optimal Control (Linear Quadratic Regulator)
                \item Optimal Noise Resistant Control (Linear Quadratic Gaussian) 
                \item Machine Learning (Approximate Dynamic Programming)
            \end{itemize}
    \end{itemize}
\end{frame}

%----------------------------------

% \subsection*{Demonstration}

% \begin{frame}{Demonstration}
%     \centering
%     \includemedia[
%      width=0.8\linewidth,
%      height=0.6\linewidth,
%      activate=pageopen,  
%      addresource=videos/Demonstration_Trim.mp4,
%      flashvars={
%       source=videos/Demonstration_Trim.mp4
%       &loop=true  % loop the video
%      }
%     ]{}{VPlayer.swf}
% \end{frame}

%----------------------------------

% \begin{frame}{Introduction}{}
% \begin{figure}
%   \centering
%   \includegraphics[scale=0.31]{figs/ipe/highLevel_wht_grn}
%   \caption{General high-level system architecture}
%   \label{fig:ProblemStatementImage}
% \end{figure}
% \end{frame}

% \begin{frame}{Introduction}{}
%         \begin{itemize}
%         \item This project will:
%             \begin{itemize}
%                 \item use a pair of 2-DOF (2-degrees-of-freedom) mechatronics platforms
%                 \item implement control algorithms on embedded system
%                 \item use mobile device for user control
%                 \item encourage research
%                 \item serve as an educational tool
%             \end{itemize}
%         \end{itemize}
% \end{frame}

% %----------------------------------

% \section{Background Study}

% \subsection{Control Techniques}

% \begin{frame}{Background Study}{Control Techniques}
%     Various control techniques have been proposed for 2-DOF helicopters such as:
%     \begin{itemize}
%         \item Sliding mode control \cite{Ahmed2010-Sliding} %cite 2-Sliding Mode Based Robust Control for 2-DOF Helicopter here
%         \item Fuzzy Logic control 
%         \cite{Chang2017-Fuzzy}
%         \cite{Kayacan2016-Fuzzy}
%         \cite{Mendez-Monroy2012-Fuzzy} 
%         %cite Fuzzy control with estimated variable sampling period for non-linear networked control systems: 2-DOF helicopter as case study here
%         \item Data-driven Adaptive Optimal Output-feedback control \cite{Gao2016-DataDriven} %cite Data-driven Adaptive Optimal Output-feedback Control of a 2-DOF Helicopter here
%         \item Decentralized discrete-time neural control \cite{Hernandez-Gonzalez2012-Decentralized} %cite Decentralized discrete-time neural control for a Quanser 2-DOF helicopter here
%     \end{itemize}
%     These control techniques employ advanced mathematics that are difficult to implement on embedded systems.
% \end{frame}

% %----------------------------------

% \subsection{Modeling a 2-DOF Helicopter}

% \begin{frame}{Background Study}{Modeling a 2-DOF Helicopter}
%     \begin{columns}
%     \column{0.5\textwidth}
%     \begin{figure}
%         \centering
%         \includegraphics[width=\textwidth]{figs/img/helicopterModel}
%         \caption{Model of a 2-DOF helicopter}
%         \label{fig:helicopterModel}
%     \end{figure}
%     \column{0.5\textwidth}
%     \begin{figure}
%       \centering 
%       \includegraphics[width=\textwidth]{figs/img/QuanserAero}
%       \caption{Quanser Aero}
%       \label{fig:QuanserAero}
%     \end{figure}    
%     \end{columns}
% \end{frame}

% \begin{frame}{Background Study}{Modeling a 2-DOF Helicopter}
%     \begin{itemize}
%         \item Characterized by fixed base
%         \begin{itemize}
%             \item Can change 2 of 3 possible orientations...
%             \begin{itemize}
%                 \item Pitch ($\theta$)
%                 \item Yaw ($\psi$)
%                 \item \emph{Not Roll}
%             \end{itemize}
%             \item and cannot change position
%             \begin{itemize}
%                 \item x direction
%                 \item y direction
%                 \item z direction
%             \end{itemize}
%         \end{itemize} 
%     \end{itemize}
% \end{frame}

% % \begin{frame}{Background Study}{Quanser Aero} 

% % \end{frame}

% \begin{frame}{Background Study}{Modeling a 2-DOF Helicopter}
%     \begin{itemize}
%         \item Motors are attached to the propellers to create thrust due to air resistance
%             \begin{itemize}
%                 \item Main - changes pitch angle
%                 \item Tail - changes yaw angle
%             \end{itemize} 
%         \item Torque due to rotation also creates a force on opposite axes 
%     \end{itemize}
% \end{frame}

% \begin{frame}{Background Study}{Modeling a 2-DOF Helicopter}
% Due to the efficiency of the Quanser Aero, we can create a linearized system model:
% \begin{align}
%   %\dot{\bf x}(t) = {\bf A}{\bf x}(t) +{\bf B}{\bf u}(t),~\mathrm{where}
%     \dot{\bf x}(t) = {\bf A}{\bf x}(t) +{\bf B}{\bf u}(t),~\mathrm{such}~\mathrm{that}
% \label{eq:stateModel}
% \end{align}  
% %
% \begin{align*}
% \begin{bmatrix}
%     \dot\theta\\
%     \dot\psi\\
%     \ddot{\theta}\\
%     \ddot{\psi}
% \end{bmatrix}&=
% %\label{eq:matrixA}
% \begin{bmatrix}
%     0 & 0 & 1 & 0 \\
%     0 & 0 & 0 & 1 \\
%     0 & -K_{sp}/J_p & -D_p/J_p & 0 \\
%     0 & 0 & 1 & -D_y/J_y 
% \end{bmatrix} 
% %\label{eq:stateMatrix} 
% \begin{bmatrix}
%     \theta\\
%     \psi\\
%     \dot{\theta}\\
%     \dot{\psi}
% \end{bmatrix}
% \\&+
% %\label{eq:matrixB}
% \begin{bmatrix}
%     0 & 0 \\
%     0 & 0 \\
%     K_{pp}/J_p & K_{py}/J_p \\
%     K_{yp}/J_y & K_{yy}/J_y 
% \end{bmatrix}
% %\label{eq:inputMatrix}
% \begin{bmatrix}
%     V_p \\
%     V_y 
% \end{bmatrix}
% %{\bf A} =  
% %\begin{bmatrix}
% %0 & 0 & 1 & 0\\
% %0 & 0 & 0 & 1\\
% %-\frac{K_{\text{sp}}}{J_p} & 0 & -\frac{D_p}{J_p} &  0\\
% %0 & 0 & 0 & -\frac{D_y}{J_y}    
% %\end{bmatrix}
% %~\text{and}~
% %  {\bf B} =
% %\begin{bmatrix}
% %0 & 0\\
% %0 & 0\\
% %\frac{K_{\text{pp}}}{J_p} & \frac{K_{\text{py}}}{J_p}\\
% %\frac{K_{\text{yp}}}{J_y} & \frac{K_{\text{yy}}}{J_y}                           
% %\end{bmatrix}
% \end{align*}
% \end{frame}

% \begin{frame}{Background Study}{Modeling a 2-DOF Helicopter}
% \begin{itemize}
%     \item $K_{sp}$ - being the stiffness of the axes
%     \item $K_{pp}$ - pitch motor thrust constant
%     \item $K_{py}$ - thrust constant acting on the pitch angle from the yaw motor
%     \item $K_{yp}$ - thrust constant acting on the yaw angle from the pitch motor
%     \item $K_{yy}$ - yaw motor thrust constant
%     \item $J_p$ - moment of inertia about pitch axis
%     \item $J_y$ - moment of inertia about yaw axis
%     \item $D_p$ - viscous damping of the pitch axis
%     \item $D_y$ - viscous damping of the yaw axis
% \end{itemize}
% \end{frame}

% %----------------------------------
% \subsection{Control Algorithm and Architecture}
% \begin{frame}{Background Study}{Control Algorithm Overview - Optimal Control}
% \begin{enumerate}
%     \item Employ state-space representation of 2-DOF helicopter:
%     \begin{align*}
%         \dot{\mathbf{x}} = \mathbf{A}\mathbf{x} + \mathbf{B}\mathbf{u}
%     \end{align*}
%     \item Use state feedback law
%     \begin{center}
%         $\mathbf{u} = -\mathbf{K}\mathbf{x}$
%     \end{center}
%     to minimize the quadratic cost function:
%     \begin{align*}
%         J(\mathbf{u}) = \int_0^\infty (\mathbf{x}^T\mathbf{Q}\mathbf{x} + \mathbf{u}^T\mathbf{R}\mathbf{u} + 2\mathbf{x}^T\mathbf{N}\mathbf{u})\mathrm{dt}
%     \end{align*}
%     \item Find the solution $\mathbf{S}$ to the Riccati equation
%     \begin{align*}
%         \mathbf{A}^T\mathbf{S}+\mathbf{SA}-(\mathbf{SB}+\mathbf{N})\mathbf{R}^{-1}(\mathbf{B}^T\mathbf{S}+\mathbf{N}^T)+\mathbf{Q}=0
%     \end{align*}    
%     \item Calculate gain, $\mathbf{K}$
%     \begin{center}
%         $\mathbf{K}=\mathbf{R}^{-1}(\mathbf{B}^T\mathbf{S}+\mathbf{N}^T)$
%     \end{center}
% \end{enumerate}
% \end{frame}
% %----------------------------------
% \begin{frame}{Background Study}{Control Algorithm Overview - Optimal Noise Resistant Control}
% \begin{itemize}
%     \item Utilizes gain calculated in LQR
%     \item Added Kalman filter to reduce external disturbances to the system
% \end{itemize} 
% \begin{figure}
%     \centering
%     \includegraphics[width=.8\textwidth,keepaspectratio=true]{figs/img/LQG_SimulinkResize.png}
%     \caption{Noise resistant 2-DOF helicopter model}
%     \label{fig:LQGModel}
% \end{figure}
% \end{frame}
% %----------------------------------
% \begin{frame}{Background Study}{Control Algorithm Overview - Machine Learning}
%     \begin{columns}
%     \column{0.5\textwidth}
%     \begin{figure}
%         \centering
%         \includegraphics[width=\textwidth]{figs/ipe/ADP_Neural_Network}
%         \caption{Neural network}
%         \label{fig:ADP_Neural_Networkl}
%     \end{figure}
%     \column{0.5\textwidth}
%     \begin{figure}
%       \centering 
%       \includegraphics[width=\textwidth]{figs/ipe/ADP_Samples}
%       \caption{Neural network sampling}
%       \label{fig:ADP_Samples}
%     \end{figure}    
%     \end{columns}
% \end{frame}
% %----------------------------------
% \begin{frame}{Background Study}{Control Architecture Overview -  P Type Controller}
% \begin{figure}
%     \centering
%     \includegraphics[width=.8\textwidth,keepaspectratio=true]{figs/ipe/P_Control}
%     \caption{Optimal P type controller [servo]}
%     \label{fig:P_Control}
% \end{figure}
% \end{frame}
% %----------------------------------
% \begin{frame}{Background Study}{Control Architecture Overview - PI Type Controller}
% \begin{figure}
%     \centering
%     \includegraphics[width=.8\textwidth,keepaspectratio=true]{figs/ipe/PI_Control}
%     \caption{Optimal PI type controller [servo]}
%     \label{fig:PI_Control}
% \end{figure}
% \end{frame}
% %----------------------------------

% \subsection{Prior Work}

% \begin{frame}{Background Study}{Prior Work}
%   \begin{itemize}
%       \item extensive modeling \& simulations
%       \item implementation of two motion control algorithms (LQR \& ADP)
%       \item one helicopter
%       %\item deployed on mobile device
%   \end{itemize}
% \end{frame}

% %----------------------------------

% % \subsection{Challenges}

% % \begin{frame}{Background Study}{Challenges}
  
% % \end{frame}

% %----------------------------------

% \section{Subsystem Level Functional Requirements}

% % put a slide with three dimensional system architecture drawing using ipe
% % another slide with explanation

% % put a slide with system block diagram

% \subsection{Block Diagram}

% \begin{frame}{Subsystem Level Functional Requirements}{Block Diagram}

% \begin{figure}
%   \centering
%   \includegraphics[scale=0.31]{figs/ipe/TCPModel}
%   \caption{Communication model}
%   \label{fig:TCPModel}
% \end{figure}

% \end{frame}

% \begin{frame}{Subsystem Level Functional Requirements}{Block Diagram} 

% \begin{figure}
%   \centering 
%   \includegraphics[scale=0.31]{figs/ipe/lowLevel}
%   \caption{Low level smart control diagram}
%   \label{fig:ProposalImage}
% \end{figure}

% \end{frame}

% %----------------------------------

% \section{Simulation}

% \subsection{Optimal Control Simulation}

% % P Controller
% \begin{frame}{Simulation}{Optimal Control Simulation (P Type Controller)}
%     \begin{figure}
%       \centering
%       \subfigure[][]{
%         \label{fig:LQR_Pos_Con}
%         \includegraphics[scale=0.38]{figs/MATLAB/LQR/P_Simulation/LQR_Pos_Con}
%       }
%       \subfigure[][]{
%         \label{fig:LQR_Volt_Con}
%         \includegraphics[scale=0.38]{figs/MATLAB/LQR/P_Simulation/LQR_Volt_Con}    
%       }  
%       \caption{Optimal control (P type controller) simulation \subref{fig:LQR_Pos_Con}~position and~\subref{fig:LQR_Volt_Con}~voltage w/ step input}
%       \label{fig:LQR_Sim_Con}
%     \end{figure}
% \end{frame}

% \begin{frame}{Simulation}{Optimal Control Simulation (P Type Controller)}
%     \begin{figure}
%       \centering 
%       \includegraphics[scale=0.5]{figs/MATLAB/LQR/P_Simulation/LQR_Error_Con}
%       \caption{Optimal control (P type controller) simulation w/ constant signal}
%       \label{fig:LQR_Error_Con}
%     \end{figure}
% \end{frame}

% % PI LQR vs LQG
% \subsection{Noise Resistant and Optimal Control Simulation}
% \begin{frame}{Simulation}{Noise Resistant and Optimal Control (PI type Controller) Simulation}
%     \begin{figure}
%       \centering
%       \subfigure[][]{
%         \label{fig:Pitch}
%         \includegraphics[scale=0.38]{figs/matlab/LQG_PIvLQR_PI_Sim/Pitch}
%       }
%       \subfigure[][]{
%         \label{fig:Yaw}
%         \includegraphics[scale=0.38]{figs/matlab/LQG_PIvLQR_PI_Sim/Yaw}    
%       }  
%       \caption{Noise resistant control vs optimal control (PI type controller) simulation \subref{fig:Pitch}~pitch~position and~\subref{fig:Yaw}~yaw~position w/ step input}
%       \label{fig:LQR_LQG_Sim_pos}
%     \end{figure}
% \end{frame}

% \begin{frame}{Simulation}{Noise Resistant and Optimal Control (PI Type Controller) Simulation}
%     \begin{figure}
%       \centering
%       \subfigure[][]{
%         \label{fig:Pitch_Volt}
%         \includegraphics[scale=0.38]{figs/matlab/LQG_PIvLQR_PI_Sim/Pitch_Volt}
%       }
%       \subfigure[][]{
%         \label{fig:Yaw_Volt}
%         \includegraphics[scale=0.38]{figs/matlab/LQG_PIvLQR_PI_Sim/Yaw_Volt}    
%       }  
%       \caption{Noise resistant control vs optimal control (PI type controller) simulation \subref{fig:Pitch_Volt}~pitch~voltage and~\subref{fig:Yaw_Volt}~yaw~voltage w/ step input}
%       \label{fig:LQR_PI_Sim_volt}
%     \end{figure}
% \end{frame}

% %% PI Controller
% %\begin{frame}{Simulation}{Optimal Control Simulation (PI Controller)}
% %    \begin{figure}
% %      \centering
% %      \subfigure[][]{
% %        \label{fig:Pitch_LQR_Sim}
% %        \includegraphics[scale=0.38]{figs/MATLAB/LQR/PI_Sim/Pitch_LQR_Sim}
% %      }
% %      \subfigure[][]{
% %        \label{fig:Yaw_LQR_Sim}
% %        \includegraphics[scale=0.38]{figs/MATLAB/LQR/PI_Sim/Yaw_LQR_Sim}    
% %      }  
% %      \caption{Optimal Control (PI Controller) Simulation \subref{fig:Pitch_LQR_Sim}~Pitch~Position and~\subref{fig:Yaw_LQR_Sim}~Yaw~Position w/ Step Input}
% %      \label{fig:LQR_PI_Sim_pos}
% %    \end{figure}
% %\end{frame}
% %
% %\begin{frame}{Simulation}{Optimal Control (PI Controller) Simulation}
% %    \begin{figure}
% %      \centering
% %      \subfigure[][]{
% %        \label{fig:Pitch_Volt_LQR_Sim}
% %        \includegraphics[scale=0.38]{figs/MATLAB/LQR/PI_Sim/Pitch_Volt_LQR_Sim}
% %      }
% %      \subfigure[][]{
% %        \label{fig:Yaw_Volt_LQR_Sim}
% %        \includegraphics[scale=0.38]{figs/MATLAB/LQR/PI_Sim/Yaw_Volt_LQR_Sim}    
% %      }  
% %      \caption{Optimal Control (PI Controller) Simulation \subref{fig:Pitch_Volt_LQR_Sim}~Pitch~Voltage and~\subref{fig:Yaw_Volt_LQR_Sim}~Yaw~Voltage w/ Step Input}
% %      \label{fig:LQR_PI_Sim_volt}
% %    \end{figure}
% %\end{frame}
% %
% %% LQG
% %\subsection{Optimal Noise Resistant Control Simulation}
% %\begin{frame}{Simulation}{Optimal Noise Resistant Control (PI Controller) Simulation}
% %    \begin{figure}
% %      \centering
% %      \subfigure[][]{
% %        \label{fig:Pitch_LQG_Sim}
% %        \includegraphics[scale=0.38]{figs/MATLAB/LQG/LQG_Sim/Pitch_LQG_Sim}
% %      }
% %      \subfigure[][]{
% %        \label{fig:Yaw_LQG_Sim}
% %        \includegraphics[scale=0.38]{figs/MATLAB/LQG/LQG_Sim/Yaw_LQG_Sim}    
% %      }  
% %      \caption{Optimal Noise Resistant Control (PI Controller) \subref{fig:Pitch_LQG_Sim}~Pitch~Position and~\subref{fig:Yaw_LQG_Sim}~Yaw~Position w/ Step Input}
% %      \label{fig:LQG_PI_Sim_pos}
% %    \end{figure}
% %\end{frame}
% %
% %\begin{frame}{Simulation}{Optimal Noise Resistant Control (PI Controller) Simulation}
% %    \begin{figure}
% %      \centering
% %      \subfigure[][]{
% %        \label{fig:Pitch_Volt_LQG_Sim}
% %        \includegraphics[scale=0.38]{figs/MATLAB/LQG/LQG_Sim/Pitch_Volt_LQG_Sim}
% %      }
% %      \subfigure[][]{
% %        \label{fig:Yaw_Volt_LQG_Sim}
% %        \includegraphics[scale=0.38]{figs/MATLAB/LQG/LQG_Sim/Yaw_Volt_LQG_Sim}    
% %      }  
% %      \caption{Optimal Noise Resistant Control Simulation (PI Controller) \subref{fig:Pitch_Volt_LQG_Sim}~Pitch~Voltage and~\subref{fig:Yaw_Volt_LQG_Sim}~Yaw~Voltage w/ Step Input}
% %      \label{fig:LQG_PI_Sim_volt}
% %    \end{figure}
% %\end{frame}

% %----------------------------------

% \section{Implementation}

% \subsection{USB}
% %LQR
% \begin{frame}{Implementation}{Optimal Control P and PI Type Controller USB}
%     \begin{figure}
%       \centering
%       \subfigure[][]{
%         \label{fig:Pitch_LQR_RMSE}
%         \includegraphics[scale=0.38]{figs/matlab/LQR_PIvLQR_P_USB/step/Pitch_LQR_RMSE}
%       }
%       \subfigure[][]{
%         \label{fig:Yaw_LQR_RMSE}
%         \includegraphics[scale=0.38]{figs/matlab/LQR_PIvLQR_P_USB/step/Yaw_LQR_RMSE}    
%       }  
%       \caption{USB implementation comparison between optimal control (P~type~controller) and optimal control (PI~type~controller) for \subref{fig:Pitch_LQR_RMSE}~pitch and \subref{fig:Yaw_LQR_RMSE}~yaw configurations w/ step input}
%       \label{fig:PvPI_USB}
%     \end{figure}
% \end{frame}
% %
% \begin{frame}{Implementation}{Optimal Control P and PI Type Controller USB}
%     \begin{figure}
%       \centering
%       \subfigure[][]{
%         \label{fig:PitchVoltage_LQR_RMSE}
%         \includegraphics[scale=0.38]{figs/matlab/LQR_PIvLQR_P_USB/step/PitchVoltage_LQR_RMSE}
%       }
%       \subfigure[][]{
%         \label{fig:YawVoltage_LQR_RMSE}
%         \includegraphics[scale=0.38]{figs/matlab/LQR_PIvLQR_P_USB/step/YawVoltage_LQR_RMSE}    
%       }  
%       \caption{USB implementation comparison between optimal control (P~type~controller) and optimal control (PI~type~controller) for \subref{fig:PitchVoltage_LQR_RMSE}~pitch and \subref{fig:YawVoltage_LQR_RMSE}~yaw voltages w/ step input}
%       \label{fig:PvPIVoltage_USB}
%     \end{figure}
% \end{frame}
% %
% \begin{frame}{Implementation}{Optimal Control P and PI Type Controller USB}
% \begin{table}
%     \centering
%     \begin{tabular}{l|l|l}
%         \toprule
%         \textbf{} & \textbf{Pitch Step} & \textbf{Yaw Step}\\
%         \toprule
%         LQR P & 3.5025 & 5.8502\\
%         LQR PI & 1.2349 & 5.5058\\
%         Improvement & 64.7437\% & 0.5408\% \\
%         \toprule
%         \textbf{} & \textbf{Pitch Square} & \textbf{Yaw Square}\\
%         \toprule
%         LQR P & 6.2819 & 20.4623\\
%         LQR PI & 6.9206 & 21.0709\\
%         Improvement & -10.1675\% & -2.9740\% \\
%         \toprule
%         \textbf{} & \textbf{Pitch Sine} & \textbf{Yaw Sine}\\
%         \toprule
%         LQR P & 4.2469 & 2.8644\\
%         LQR PI & 1.3383 & 1.7852\\
%         Improvement & 68.4872\% & 63.2998\% \\
%     \end{tabular}
%     \caption{Root mean squared error}
%     \label{tab:RMSE}
% \end{table}
% \end{frame}
% %\begin{frame}{Implementation}{Optimal Control P and PI Controller USB}
% %\begin{table}
% %    \centering
% %    \begin{tabular}{l|l|l|l|l|l|l}
% %        \toprule
% %        \textbf{0} & \textbf{Pitch Step} & \textbf{Yaw Step} & \textbf{Pitch Square} & \textbf{Yaw Square} & \textbf{Pitch Sine} & \textbf{Yaw Sine}\\
% %        \toprule
% %        LQR P & 3.5025 & 5.8502 & 6.2819 & 20.4623 & 4.2469 & 2.8644\\
% %        LQR PI & 1.2349 & 5.5058 & 6.9206 & 21.0709 & 1.3383 & 1.7852
% %        \bottomrule
% %    \end{tabular}
% %    \caption{Root Mean Squared Error}
% %    \label{tab:RMSE}
% %\end{table}
% %\end{frame}
% %Machine vs LQR
% \begin{frame}{Implementation}{Machine Learning and Optimal Control (P Type Controller) USB}
%     \begin{figure}
%       \centering
%       \subfigure[][]{
%         \label{fig:Pitch_LQRvADP_USB}
%         \includegraphics[scale=0.38]{figs/matlab/ADPvLQR_P_USB/Pitch_ADP_LQR}
%       }
%       \subfigure[][]{
%         \label{fig:Yaw_LQRvADP_USB}
%         \includegraphics[scale=0.38]{figs/matlab/ADPvLQR_P_USB/Yaw_ADP_LQR}    
%       }  
%       \caption{USB implementation comparison between machine learning and optimal control (P type controller) for \subref{fig:Pitch_LQRvADP_USB}~pitch and \subref{fig:Yaw_LQRvADP_USB}~yaw orientations w/ step input}
%       \label{fig:LQRvADP_USB}
%     \end{figure}
% \end{frame}
% \begin{frame}{Implementation}{Machine Learning and Optimal Control (P Type Controller) USB}
% \begin{table}
%     \centering
%     \begin{tabular}{l|l|l}
%         \toprule
%         \textbf{} & \textbf{Pitch Step} & \textbf{Yaw Step}\\
%         \toprule
%         ADP P & 1.3067 & 6.1991\\
%         LQR P & 3.5025 & 5.8502\\
%         Improvement & 62.6923\% & -5.9638\% \\
%         \toprule
%         \textbf{} & \textbf{Pitch Square} & \textbf{Yaw Square}\\
%         \toprule
%         ADP P & 6.5790 & 21.1923\\
%         LQR P & 6.2819 & 20.4623\\
%         Improvement & -4.7294\% & -0.3567\% \\
%         \toprule
%         \textbf{} & \textbf{Pitch Sine} & \textbf{Yaw Sine}\\
%         \toprule
%         ADP P & 2.1877 & 3.6307\\
%         LQR P & 4.2469 & 2.8644\\
%         Improvement & 48.4871\% & -26.7525\% \\
%     \end{tabular}
%     \caption{Root mean squared error}
%     \label{tab:RMSE2}
% \end{table}
% \end{frame}
% \subsection{Android}
% %LQR
% \begin{frame}{Implementation}{Optimal Control (P Type Controller) via Android}
%     \begin{figure}
%       \centering
%       \subfigure[][]{
%         \label{fig:LQR_Pitchpos}
%         \includegraphics[scale=0.38]{figs/matlab/LQR/P_Android/LQR_Pitchpos}
%       }
%       \subfigure[][]{
%         \label{fig:LQR_Yawpos}
%         \includegraphics[scale=0.38]{figs/matlab/LQR/P_Android/LQR_Yawpos}    
%       }  
%       \caption{Optimal control (P type controller) \subref{fig:LQR_Pitchpos}~pitch~position and~\subref{fig:LQR_Yawpos}~yaw~position w/ input from mobile phone}
%       \label{fig:LQR_pos}
%     \end{figure}
% \end{frame}
% %
% \begin{frame}{Implementation}{Optimal Control (P Type Controller) via Android}
%     \begin{figure}
%       \centering
%       \subfigure[][]{
%         \label{fig:LQR_PitchVolt}
%         \includegraphics[scale=0.38]{figs/matlab/LQR/P_Android/LQR_PitchVolt}
%       }
%       \subfigure[][]{
%         \label{fig:LQR_YawVolt}
%         \includegraphics[scale=0.38]{figs/matlab/LQR/P_Android/LQR_YawVolt}    
%       }  
%       \caption{Optimal control (P type controller) \subref{fig:LQR_PitchVolt}~pitch~voltage and~\subref{fig:LQR_YawVolt}~yaw~voltage w/ input from mobile phone}
%       \label{fig:LQR_Volt}
%     \end{figure}
% \end{frame}
% %ADP
% \begin{frame}{Implementation}{Machine Learning via Android}
%     \begin{figure}
%       \centering
%       \subfigure[][]{
%         \label{fig:ADP_Pitch_Wireless}
%         \includegraphics[scale=0.38]{figs/matlab/ADP/Android/ADP_Pitch_Wireless}
%       }
%       \subfigure[][]{
%         \label{fig:ADP_Yaw_Wireless}
%         \includegraphics[scale=0.38]{figs/matlab/ADP/Android/ADP_Yaw_Wireless}    
%       }  
%       \caption{Machine learning \subref{fig:ADP_Pitch_Wireless}~pitch~position and~\subref{fig:ADP_Yaw_Wireless}~yaw~position w/ input from mobile phone}
%       \label{fig:ADP_Wireless}
%     \end{figure}
% \end{frame}
% %
% \begin{frame}{Implementation}{Machine Learning via Android}
%     \begin{figure}
%       \centering
%       \subfigure[][]{
%         \label{fig:ADP_Pitch_Volt_Wireless}
%         \includegraphics[scale=0.38]{figs/matlab/ADP/Android/ADP_Pitch_Volt_Wireless}
%       }
%       \subfigure[][]{
%         \label{fig:ADP_Yaw_Volt_Wireless}
%         \includegraphics[scale=0.38]{figs/matlab/ADP/Android/ADP_Yaw_Volt_Wireless}    
%       }  
%       \caption{Machine learning \subref{fig:ADP_Pitch_Volt_Wireless}~pitch~voltage and~\subref{fig:ADP_Yaw_Volt_Wireless}~yaw~voltage w/ input from mobile phone}
%       \label{fig:ADP_Volt_Wireless}
%     \end{figure}
% \end{frame}

% %----------------------------------
% %
% %\section{Parts List}
% %
% %\begin{frame}{Parts List}{}
% %
% %  \begin{itemize}
% %      \item Hardware
% %      \begin{itemize}
% %        \item Two Quanser Aeros
% %            \begin{itemize}
% %                \item Q-flex2 Embedded Panel
% %            \end{itemize}
% %        \item Two Single Board Computers (Raspberry Pi 3 Model B)
% %        \item Android Smart-phone or Tablet\\
% %        (Note that Apple devices could also be used, however modifications are needed)
% %      \end{itemize}
% %      \item Software
% %      \begin{itemize}
% %          \item MATLAB \& Simulink
% %          \begin{itemize}
% %            \item Raspberry Pi Support Package
% %            \item Android Support Package
% %          \end{itemize}
% %          \item Quanser Real-Time Control (QUARC)
% %          % "Quanser’s QUARC™ software adds powerful tools and capabilities to Simulink® that make the development and deployment of sophisticated real-time mechatronics and control applications easier. QUARC™ generates real-time code directly from Simulink-designed controllers and runs it in real-time on the Windows target – all without digital signal processing or without writing a single line of code."
% %      \end{itemize}
% %  \end{itemize}
% %\end{frame}

% %----------------------------------



% \section{Future Directions}

% \begin{frame}{Future Directions}{} 
%   \begin{itemize}
%     \item Discretization of System   
%     \item Digital Compass
%     \item Enhanced Smart Control
% 	\vskip .2cm
% 	\begin{figure}
% 	  \centering
% 	  \includegraphics[scale=0.28]{figs/ipe/smartAlg}
% 	  \caption{Enhanced smart control}
% 	  \label{fig:smartAlg}
% 	\end{figure}
% 	%\vskip -.4cm
%     \item Implmentation on 6-DOF Helicopter
%   \end{itemize}
% \end{frame}

% % You can reveal the parts of a slide one at a time
% % with the \pause command:
% %\begin{frame}{Second Slide Title}
% %  \begin{itemize}
% %  \item {
% %    First item.
% %    \pause % The slide will pause after showing the first item
% %  }
%   %\item {   
%   %  Second item.
%  % }
%   % You can also specify when the content should appear
%   % by using <n->:
%  % \item<3-> {
%  %   Third item.
%  % }
% %  \item<4-> {
% %    Fourth item.
%  % }
%   % or you can use the \uncover command to reveal general
%   % content (not just \items):
% %  \item<5-> {
% %    Fifth item. \uncover<6->{Extra text in the fifth item.}
% %  }
% %  \end{itemize}
% %\end{frame}

% %\section{Second Main Section}

% % \subsection{Another Subsection}

% % \begin{frame}{Blocks}
% % \begin{block}{Block Title}
% % You can also highlight sections of your %presentation in a block, with it's own %title
% % \end{block}
% % \begin{theorem}
% % There are separate environments for %theorems, examples, definitions and proofs.
% % \end{theorem}
% % \begin{example}
% % Here is an example of an example block.
% % \end{example}
% % \end{frame}

% % Placing a * after \section means it will not show in the
% % outline or table of contents.
% \section*{Summary}
% \begin{frame}{Summary} %Glenn
%   \begin{itemize}
%     \item Embedded implementation of control algorithms
%     \item Mobile interface
%     \item PI type control improves steady-state error
%     \item Machine Learning is best when system parameters are unknown or time-varient
%     %\item\todo[inline]{Add table for RMSE?}
%   \end{itemize}
  
% %   \begin{itemize}
% %   \item Outlook
% %     \begin{itemize}
% %     \item Implementation using two more motion control algorithms (LQG \& ADP)
% %     \item Development of a test plan to compare algorithms
% %     \end{itemize}
% %   \end{itemize}
% \end{frame}
% % \begin{frame}{Summary}
% %   \begin{itemize}
% %   \item
% %     The \alert{first main message} of your talk in one or two lines.
% %   \item
% %     The \alert{second main message} of your talk in one or two lines.
% %   \item
% %     Perhaps a \alert{third message}, but not more than that.
% %   \end{itemize}
  
% %   \begin{itemize}
% %   \item
% %     Outlook
% %     \begin{itemize}
% %     \item
% %       Something you haven't solved.
% %     \item
% %       Something else you haven't solved.
% %     \end{itemize}
% %   \end{itemize}
% % \end{frame}

% \section*{Acknowledgement}
%   \begin{frame}{Acknowledgement} 
%     \centering
%     Special Thanks to Andrew Fandel, Anthony Birge, and Dr. Suruz Miah for their work with Machine Learning on a 2-DOF Helicopter
%   \end{frame}

% All of the following is optional and typically not needed. 
\appendix
\section<presentation>*{\appendixname}
\subsection<presentation>*{For Further Reading}

\begin{frame}[allowframebreaks]
  \frametitle<presentation>{For Further Reading}
  \bibliographystyle{IEEEtran}
  \bibliography{bib/references,bib/refsHelicopter}
  
%   \begin{thebibliography}{10}
    
%   \beamertemplatebookbibitems
%   % Start with overview books.
    
    
%   \beamertemplatearticlebibitems
 
%   % Followed by interesting articles. Keep the list short. 
    
%     \bibitem{Article1}
% M. Hernandez-Gonzalez, A.Y. Alanis, E.A. Hernandez-Vargas,
% \emph{Decentralized discrete-time neural control for a Quanser 2-DOF helicopter }.
% 2012

% \bibitem{Article2}
% F. Dos Santos Barbosa,
% \emph{4DOF quadcopter: development, modeling and control }.
% 2017

% \bibitem{Article3}
% G.G. Neto, F.S. Barbosa, B.A. Ang\'{e}lico,
% \emph{2-DOF helicopter controlling by pole-placements }.

% \bibitem{Article4}
% Weinan Gao and Zhong-Ping Jiang
% \emph{Data-driven Adaptive Optimal Output-feedback Control of a 2-DOF Helicopter }.

% \bibitem{Article5}
% P. Méndez-Monroy, H. Benítez-Pérez,
% \emph{Fuzzy Control with Estimated Variable Sampling Period for Non-Linear Networked Control Systems: 2-DOF Helicopter as Case Study }. 
% Transactions of the Institute of Measurement \& Control 34, no. 7 (October 15, 2012): 802–14. 
    
%     \bibitem{Article6}
%     Q. Ahmed and A.I.Bhatti and S.Iqbal and I.H. Kazmi
%     \emph{2-Sliding Mode Based Robust Control for 2-DOF Helicopter}.
    
% %   \bibitem{Someone2000}
% %     S.~Someone.
% %     \newblock On this and that.
% %     \newblock {\em Journal of This and That}, 2(1):50--100,
% %     2000.
%   \end{thebibliography}
\end{frame}

\end{document}



%%% Local Variables:
%%% mode: latex
%%% TeX-master: t
%%% End:
